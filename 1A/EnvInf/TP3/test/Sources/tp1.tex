%%%%%%%%%%%%%%%%%%%%%%%%%%%%%%%%%%%%%%%%%%%%%%%%%%%%%%%%%%%%%%%%%%%%%%%%%%%%%%%%
%
%  Auteur :  Ronan Guivarc'h             
%  Création :  01/09/04              
%  Modifications : Ronan Guivarc'h (mozilla -> firefox 22/09/05)
%
%%%%%%%%%%%%%%%%%%%%%%%%%%%%%%%%%%%%%%%%%%%%%%%%%%%%%%%%%%%%%%%%%%%%%%%%%%%%%%%%

\documentclass[a4paper,11pt]{article}

\usepackage{../tp_utf8}

%----- commandes locales %------------------------------------------------------
\newcommand{\netscape}{\textsc{Firefox}\xspace}
\newcommand{\openwin}{\textsc{Java Desktop System}\xspace}
\newcommand{\term}{\textsc{Terminal}\xspace}
\newcommand{\unix}{\textsc{unix}\xspace}
\newcommand{\linux}{\textsc{linux}\xspace}
\newcommand{\ubuntu}{\textsc{ubuntu}\xspace}
\newcommand{\gnome}{\textsc{xfce}\xspace}
\newcommand{\shell}{\textsc{shell}\xspace}

%====[ PAGE DE TITRE ]==========================================================
\title{Prise de contact}
\numero{TP n° 1}
\date{2013--2014}
\discipline{Utilisation des Ordinateurs}
\promotion{ENSEEIHT --- 1\iere Année Informatique}


%====[ DEBUT DU DOCUMENT ]======================================================

\begin{document}

%---------------%------------------------------ corps du document

\maketitle

\section*{Objectifs}

Les objectifs de ce premier rendez-vous officiel avec la machine sont :

\begin{enumerate}
  \item apprendre à connaître
        le système d'exploitation \linux avec une distribution \ubuntu
        et l'environnement \gnome,
  \item se familiariser avec la notion de compte, de \term et
        d'interpréteur \shell,
  \item manipuler quelques commandes de base,
  \item lancer le navigateur \netscape et consulter son mail par
        l'intermédiaire de l'outil de messagerie \textsc{Webmail}.
\end{enumerate}

\section*{Recommandations préliminaires}

\begin{itemize}
  \item Ne \textbf{jamais} éteindre les PC mais
        \textbf{toujours} penser à se déconnecter,
  \item Il est nécessaire de :
        \begin{itemize}
          \item se sentir responsable de \textit{son} compte,
          \item se sentir responsable du matériel commun,
          \item ne pas confondre salle d'ordinateurs et salle de jeux vidéo.
        \end{itemize} 
  \item Il est interdit de :
        \begin{itemize}
          \item manger, boire, fumer, ... dans les salles des machines
                (l'interdiction de fumer s'applique d'ailleurs à tout le
                bâtiment),
          \item toucher les interrupteurs (marche/arrêt),
          \item installer des programmes que vous n'avez pas écrits vous-même
                ou qui n'ont pas été fournis par les enseignants ou les
                ingénieurs systèmes.
        \end{itemize}
\end{itemize}


\section{Connexion / Déconnexion}

\subsection{Connexion}

Chaque utilisateur (\textit{user}) d'un PC a un nom de compte
qui permet son identification par la machine et un mot de passe qui évite la
connexion de personnes non autorisées.

Le nom de votre compte est {\sl pnom} où {\sl p} est la première lettre de
votre prénom et {\sl nom} est votre nom (ou une partie de votre nom). 
Ces sigles permettent de vous associer le même compte 
pendant toute votre scolarité.
Ce nom de compte, ainsi que le mot de passe associé, vous ont été
donnés lors de votre inscription (si vous les avez oubliés, voyez
avec votre enseignant de TP).

Pour travailler sur un PC (se connecter ou se \emph{loguer}), il suffit
\begin{itemize}
  \item de réveiller l'écran (appuyer sur \RET, touche marquée « \cmd{Entrée} »
       ou déplacer la souris),
  \item de cliquer sur l'icône \fbox{\cmd{Se connecter}}
  \item d'entrer son nom de compte, 
  \item taper son mot de passe (qui n'apparaît pas en clair à l'écran).
\end{itemize}

% \begin{commandes}
%    login: \underline{pnom\RET}
% \\ password:  \hspace{4em}\textnormal{Bien sûr, le mot de passe n'apparaît pas
% sur l'écran !}
% \end{commandes}

Les machines étant en réseau, vous pouvez vous connecter sur n'importe quelle
machine accessible aux élèves de 1\iere année.

%Pour travailler sous l'environnement \gnome, cliquez sur \textsc{Options},
%puis \textsc{Session} et enfin \textsc{Java Desktop System, version 3}
%avant de valider votre connexion.

Lorsque le nom et le mot de passe sont saisis sans erreur, vous êtes connecté
et l'environnement utilisateur multifenêtré \gnome est lancé. 

\subsection{Déconnexion}

Pour se déconnecter, il vous faut cliquer sur l'icône 
\includegraphics{system-shutdown-panel}
en haut à droite de la barre horizontale.

\begin{danger}
  \textbf{Ne jamais utiliser l'interrupteur marche/arrêt physique sur
  l'ordinateur !}
\end{danger}

\begin{maw}
\begin{enumerate}
  \item connectez-vous,
  \item déconnectez-vous, puis reconnectez-vous.
\end{enumerate}
\end{maw}

\section{L'environnement XFCE}

\subsection{Quelques mots}

\gnome est un environnement multi-fenêtres et multi-écrans. Vous découvrirez
petit à petit nombre de ses fonctionnalités.

\subsection{Lancement d'un \term}

Une application que vous utiliserez fréquemment (cf la partie \ref{shell}) est
un \term. Pour l'exécuter, il faut cliquer sur l'icône en forme
d'écran noir qui apparaît lorsque vous placer le curseur de la souris sur le
bord bas de l'écran. 

%sur la barre de menu du haut\footnote{Si cette icône n'est pas
%présente, rajoutez-là en glissant le raccourci \textsc{Terminal} du Menu
%(\cmd{Applications} \(\rightarrow\) \cmd{Accessoires}) sur la barre de menu.}
%ou d'appuyer conjointement sur les 3 touches \cmd{"Ctrl" + "Alt" +
%"t"}\footnote{Un premier raccourci clavier.}.

Une fenêtre s'ouvre dans laquelle vous pouvez taper des commandes (cf. la
partie \ref{shell}) à condition de positionner la souris sur cette fenêtre
et de la rendre ``active". Par défaut, il faut cliquer sur la fenêtre pour la
rendre active, mais vous pourrez avec le temps personnaliser ces comportements
et actions.

\subsection{Fermeture d'un \term}
\label{fermeture}

Pour arrêter une application quelconque, nous vous conseillons
d'utiliser la commande adéquate dans le menu de l'application.
Généralement la commande d'arrêt se trouve dans le menu \textsc{Fichier}
(ou \textsc{File}).

Pour l'application \term, c'est dans le menu \textsc{Fichier}, la commande
\textsc{fermer la fenêtre}.

Cette procédure est valable quelle que soit l'application et permet entre autre
une sauvegarde des informations liées à l'application que l'on termine.

\section{L'interpréteur \shell à partir d'un terminal et la gestion de fichiers}

\subsection{Le minimum à savoir sur le \shell}

Quand vous ouvrez un terminal, celui-ci exécute un interpréteur de commande
\unix que l'on appelle un \shell\footnote{Le sujet du TP3 fournit quelques
détails sur le fonctionnement de l'interpréteur de commandes}.

Dans ce terminal, vous allez taper des commandes que le \shell
va interpréter. Une commande ne sera interprétée par le \shell que lorsque
vous l'aurez validée en tapant sur {\RET} (touche marquée \cmd{Entrée}).

Si l'interprétation est correcte, cela va déclencher l'appel
\begin{itemize}
  \item soit, à une commande interne au \shell (affectation de variables
        d'environnement, \cmd{cd}),
  \item soit, à un programme installé ou application (\cmd{ls}, \cmd{evince},
        \netscape),
  \item soit, à un programme que vous aurez écrit.
\end{itemize}

Sinon un message d'erreur s'affichera.

\subsection{L'organisation de votre répertoire UNIX}
\label{shell}

Les systèmes d'exploitation actuels (dont \unix) organisent les fichiers
sous forme d'une arborescence. Cette arborescence a une racine,
des noeuds non terminaux (\textbf{répertoires}) et des noeuds terminaux
(\textbf{fichiers}) ; la racine de cette hiérarchie
est un répertoire particulier dénoté \verb+/+ sous \unix.

Un noeud de l'arborescence est désigné par le chemin (suite de noeuds)
reliant la racine à ce noeud. Par exemple \verb+/a/b/c+ désigne le fichier
\verb+c+, situé dans le répertoire \verb+b+, lui même situé dans le répertoire
\verb+a+, situé sous la racine.  \verb+/a/b/c+ est le \textbf{chemin absolu} du
fichier \verb+c+.

Afin d'alléger la désignation des fichiers, il est possible de définir un
préfixe qui sera sytématiquement ajouté aux noms de fichiers ne commençant pas
par \verb+/+. Ce préfixe est appelé \textbf{répertoire courant}.

Votre \textbf{répertoire de connexion} est le répertoire à partir duquel vous allez
travailler ; il fait partie de cette arborescence. Nous utiliserons aussi le
terme de \textbf{votre racine} ou \textbf{votre home} (en franglais !)
pour désigner ce répertoire. Il est aussi désigné par le caractère
\textasciitilde (tilde).

%Quand vous ouvrez un \shell, le répertoire courant est votre répertoire de
%connexion.

\subsection{Quelques commandes pour manipuler votre répertoire}

Nous allons découvrir quelques commandes indispensables pour gérer votre
arborescence.

\begin{itemize}
  \item \textbf{ls}, pour \emph{LiSt}, permet de lister le contenu d'un
        répertoire,
  \item \textbf{mkdir}, pour \emph{MaKe DIRectory}, crée un nouveau
        répertoire (\textit{directory} en anglais). Elle permet ainsi
        d'augmenter la profondeur de l'arborescence,
  \item \textbf{pwd}, pour \emph{Print Working Directory}, affiche le nom du
        répertoire de travail ou répertoire courant,
  \item \textbf{cd}, pour \emph{Change Directory}, permet de changer le
        répertoire courant,
  \item \textbf{man}, pour\emph{MANuel}, affiche à l'écran le manuel des
        commandes répertoriées, i.e. comment utiliser ces commandes.
\end{itemize}

\begin{maw}[Exemples et exercices]

\

Ouvrez un terminal et tapez la suite de commandes suivantes en prenant le
temps d'analyser leur résultat (le texte délimité par (* ... *) est du
commentaire et ne doit pas être tapé !)~:

\begin{commandes}
   pwd     \camlcomment{pour vérifier que l'on commence à votre
                    répertoire de connexion ; il doit être de la forme
                    /home/pnom}\\
   ls      \camlcomment{liste le contenu de votre répertoire de
                    connexion} \\
   mkdir ProgFonc   \camlcomment{crée le répertoire de nom ProgFonc à votre
                    répertoire de connexion}\\
   ls      \camlcomment{ProgFonc doit apparaître dans la liste}\\
   mkdir UtilisationDesOrdinateurs\\
   ls\\
   cd ProgFonc \camlcomment{descend dans le répertoire ProgFonc qui devient le
                    répertoire courant}\\
   ls      \camlcomment{le contenu du répertoire ProgFonc est vide}\\
   cd      \camlcomment{permet de remonter à votre répertoire
                     de connexion}\\
   cd UtilisationDesOrdinateurs    \camlcomment{descend dans le répertoire
   UtilisationDesOrdinateurs}\\
   mkdir TP      \camlcomment{crée le répertoire TP dans le répertoire
   UtilisationDesOrdinateurs}\\
   ls\\
   cd TP         \camlcomment{descend dans le répertoire TP}\\
   pwd           \camlcomment{affiche l'endroit où on se trouve}\\
   cd ..         \camlcomment{remonte au père du répertoire courant}\\
   pwd\\
   cd            \camlcomment{on se place au répertoire de connexion}\\
\end{commandes}
\end{maw}

\begin{maw}[Exercice]
accédez au manuel des commandes que vous venez de taper
\begin{commandes}
   man pwd       \camlcomment{affiche le manuel de la commande pwd}\\
   man ls        \camlcomment{affiche le manuel de la commande ls}\\
   man mkdir     \camlcomment{affiche le manuel de la commande mkdir}\\
%   man cd        \camlcomment{affiche le manuel de la commande cd}\\
\end{commandes}
\end{maw}

Commandes utiles pour naviguer dans le manuel~:
\begin{itemize}
  \item \textbf{q} (quit) pour quitter le manuel
  \item \textbf{f} (forward) pour descendre d'une page
  \item \textbf{b} (back) pour remonter d'une page
  \item \textbf{/texte\_recherché} pour rechercher un motif
        \begin{itemize}
          \item[\(\rightarrow\)] \textbf{n} (next) pour aller à la prochaine occurence du motif
        \end{itemize}
\end{itemize}
\vspace{0.3cm}
\begin{maw}[Exercice]
dessinez l'arborescence que vous venez de créer.
\end{maw}

\begin{maw}[Conseil]
Tapez des commandes ça prend du temps (surtout quand les noms sont longs !)
\begin{itemize}
  \item Utilisez les flèches \textbf{haut/bas} pour parcourir l'historique des
        commandes déjà tapées,
  \item Utilisez la touche \textbf{Tab} (tabulation) pour compléter un nom de
        commande, un nom de répertoire, un nom de fichier.
\end{itemize}
\end{maw}

\newpage
\section{Le navigateur \netscape}

\subsection{Comment exécuter \netscape}

%\paragraph{À partir de la barre de menu \openwin}
%
%\ 
%
%Cliquez avec la souris sur l'icône \textbf{ À COMPLETER}

\paragraph{À partir d'un \shell}

\

Dans un terminal, tapez
\begin{commandes}
firefox
\end{commandes}

Cela va exécuter l'application \netscape qui sera votre navigateur
sous \unix.

%Lors de la première exécution, une fenêtre de configuration s'ouvre,
%choisissez de faire de \netscape votre navigateur par défaut.

% \subsection{Comment arrêter \netscape}
% 
% Comme il a été signalé précédemment (cf \ref{fermeture}), une application
% se ferme proprement par une action adéquate.
% 
% %Pour \netscape, dans le menu \textsc{Fichier}, c'est l'action
% %\textsc{Quitter}.
% Pour \netscape, dans le menu \textsc{File}, c'est l'action
% \textsc{Quit}.
% 
% Ceci permettra d'arrêter \textbf{proprement} l'application \netscape et
% évitera d'avoir des surprises la prochaine fois que vous l'exécuterez.
% 
\subsection{Application en tâche de fond}

Si vous avez exécuté \netscape à partir d'un \shell comme décrit précédemment,
vous remarquez que ce \shell est en attente de la terminaison de
l'application. Vous ne pouvez pas l'utiliser pour taper d'autres commandes
tant que \netscape fonctionne.

Deux solutions~:

\begin{enumerate}
  \item soit, vous arrêtez le programme \netscape et vous le relancez en le
        mettant en \textit{tâche de fond} en rajoutant à la commande un \&~:
        \begin{commandes}
          firefox \&
        \end{commandes}
        Cela va ``détacher" l'application du \shell et permettra de saisir de
        nouvelles commandes dans le \shell.
  \item soit, dans le terminal où vous avez exécuté \netscape, vous appuyez en
        même temps sur les deux touches Control et z. Cette séquence de
        touches va suspendre (et non arrêter) \netscape. À ce moment vous ne
        pouvez plus utiliser \netscape mais le \shell est à nouveau
        disponible. Pour réactiver \netscape, tapez dans le terminal la
        commande
        \begin{commandes}
          bg
        \end{commandes}
        Elle met l'application en tâche de fond (\cmd{bg} est pour
        \emph{BackGround}).

        Vous pouvez maintenant utiliser à la fois le \shell et l'application
        \netscape.
\end{enumerate}

\begin{danger}
  \textbf{La mise en tâche de fond est à éviter pour les commandes non
  graphiques (s'exécutant dans un terminal).}
\end{danger}

\subsection{L'outil de messagerie \textsc{Webmail}}

Dans la zone d'adresse web de \netscape, tapez l'adresse suivante

\begin{verbatim}
https://rcmail.inp-toulouse.fr
\end{verbatim}

% Cette adresse est une adresse sécurisée et la sécurité se fait par
% l'intermédiaire d'un certificat.  Lors du premier accès, afin de ne plus avoir
% à vérifier le certificat par la suite, choisissez \textbf{Accept this
% certificate permanently} (premier choix proposé).
% 
% Une deuxième fenêtre de sécurité s'ouvre, vous indiquant que vous allez
% consulter une page sécurisée. Cliquez \textsc{OK}.

Cette adresse vous amène sur la page d'accueil de la messagerie
\textsc{Webmail}. Par l'intermédiaire de cet outil, vous pourrez
accéder à votre mail quel que soit l'ordinateur (même en dehors de l'enseeiht).

La page d'accueil vous demande un login et un password : ils sont identiques
par défaut à votre login et à votre password de connexion.

% Il y a possibilité de faire mémoriser à \netscape ces informations.
% Nous vous déconseillons cependant de le faire (choix \textbf{Jamais pour ce
% site} proposé par \netscape) et ceci pour prendre de bonnes habitudes de
% sécurité.

Une fois reconnu par \textsc{Webmail}, vous avez accès aux mails que vous avez
reçus.
% ; normalement vous devez avoir dans votre boîte de réception au moins 
% un mail de votre enseignant de TP.

\begin{maw}[Exercices]

\begin{enumerate}
% \item répondez à votre enseignant pour acquitter la réception de son message,
  \item envoyez-vous un message,
  \item envoyez un message à un de vos camarades de TP,
%  \item le message de votre enseignant contient un fichier texte attaché :
%        sauvegardez-le dans votre répertoire de connexion. Vous pouvez le
%        visualiser grâce à la commande \cmd{more}~:
%        \begin{commandes}
%        more tolkien.txt
%        \end{commandes}
  \item envoyer un mail à votre enseignant en précisant votre groupe de TP.

       \

\end{enumerate}
\end{maw}

\subsection{Comment sauvegarder un \textit{Favori, Marque-page (Bookmark)}}

Plutôt que de retaper à chaque fois l'adresse d'accueil de \textsc{Webmail},
il est fortement conseillé de la sauvegarder dans les \textit{Favoris} de
\netscape. Pour cela, dans le menu %\textsc{Bookmarks},
\textsc{Marque-pages}, 
choisissez l'action %\textsc{bookmark this page}
\textsc{Marquez cette page}.
%et \textsc{OK}.
Vous pourrez ensuite organiser vos favoris à votre convenance.

\subsection{Autre façon d'exécuter \netscape}

Vous pouvez exécuter le navigateur web directement en cliquant sur l'icône
correspondante dans la barre de menu du haut (petit renard orange entourant un
globe bleu).

\section{Les Documents "Utilisation des Ordinateurs"}

Sur le \textsc{Portail} de l'ENSEEIHT se trouvent
des pages liées aux enseignements. En particulier, le module Utilisation des
Ordinateurs~:

\begin{enumerate}
  \item allez sur la page \verb+ http://www.enseeiht.fr+,
  \item Cliquez sur \verb+Portail ENT+ dans le menu \textbf{accès rapide} à
        droite,
  \item acceptez le certificat bien que \netscape indique un échec de la
        connexion sécurisée (acceptez d'ajouter une exception, obtenez le
        certificat et confirmez l'exception de sécurité),
  \item identifiez-vous (login, password),
  \item si c'est votre première connexion, complétez votre profil.
\end{enumerate}

Vous êtes maintenant dans le \textsc{Portail ENSEEIHT} à partir duquel vous
pouvez par exemple accéder aux documents présents sur votre compte à
l'ENSEEIHT.

Intéressons-nous à l'onglet \textsc{Mes Cours}.

\begin{enumerate}
  \item cliquez sur l'onglet \textsc{Mes Cours},
  \item suivez le chemin ``Cours du département
        Informatique", ``1ère année", ``Utilisation des Ordinateurs",
  \item inscrivez-vous.
\end{enumerate}

Le cours Utilisation des Ordinateurs  est défini par un outil de gestion de cours qui
se nomme \textsc{Moodle} que vous aurez l'occasion de manipuler très souvent
au cours de votre scolarité.

Pour le cours présent, vous avez à disposition les sujets de TP ainsi que les
fichiers utiles à leur réalisation.

Pour les prochaines séances, le sujet ne sera pas distribué sous forme
papier et il faudra le récupérer sur cette page.

Il vous est donc conseillé de sauvegarder aussi la page du \textsc{Portail}
dans vos favoris.

Cette procédure de connexion au Portail ou au Webmail pourra être reproduite
de partout (chez vous, cyber-café, autre école, ...). Cela permet d'avoir un
environnement ``nomade".


\section{CQFAR (Ce Qu'il Faut Avoir Retenu)}

À la fin de ce premier TP vous devez~:

\begin{enumerate}
  \item savoir vous connecter et vous déconnecter,
  \item être familier avec les termes : système d'exploitation \linux,
        environnement graphique \gnome, interpréteur \shell,
  \item pouvoir ouvrir un terminal,
  \item savoir utiliser quelques commandes \cmd{man}, \cmd{ls}, \cmd{mkdir},
        \cmd{cd}, \cmd{pwd}, \cmd{more}, \cmd{bg},
  \item savoir ouvrir le navigateur \netscape en tâche de fond,
  \item être capable de lire votre mail et d'envoyer des messages,
  \item se connecter au Portail et accéder aux documents de cours.
  %\item savoir gérer la queue d'impression par \cmd{lpstat} et \cmd{cancel}.
\end{enumerate}

Si l'un de ces objectifs n'est pas atteint, entraînez-vous d'ici le prochain
TP et n'hésitez pas à contacter votre enseignant en cas de blocage
persistant. La procédure la plus simple est de le contacter par mail.

%---------------%------------------------------ fin du document
\end{document}
