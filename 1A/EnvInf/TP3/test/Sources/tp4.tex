\documentclass[11pt,a4paper]{article}

\usepackage{../tp_utf8}

\title{Droits, \cmd{ssh} et variables d'environnement}
\date{2013--2014}
\numero{TP n° 4}
\promotion{1\iere année Informatique et Mathématiques Appliquées}
\discipline{Utilisation des Ordinateurs}

\newcommand{\unix}{\textsc{Unix}}
\newcommand{\shell}{\textsc{Shell}}

\begin{document}

\maketitle

\section*{Objectifs}

\begin{itemize}
  \item notion de processus, commandes \cmd{ps}, \cmd{kill},
  \item comprendre et manipuler la notion de droits sous \textsc{Unix},
  \item se connecter à distance et lancer des applications sur d'autres
        ordinateurs,
  \item connaître les variables globales d'environnement les plus importantes.
\end{itemize}

\section{Processus Unix}

Sous \unix, l'activité d'exécution d'un programme par le processeur est
représentée par un processus. 
Le contrôle de ces traitements en cours d'exécution est possible au moyen~:

\begin{itemize}
  \item de commandes \unix : \cmd{ps} (pour lister les processus existants),
        \cmd{kill} (pour transmettre des signaux à un processus, en
        particulier tuer un processus),
  \item de commandes spécifiques au \shell{} : \cmd{jobs}, \cmd{fg},
        \cmd{bg} \ldots
\end{itemize}

\subsection{La commande \cmd{ps}}

La commande \cmd{ps} permet de donner des informations à propos des processus 
existants, notamment : 
\begin{itemize}
  \item \cmd{PID} : identifiant (unique) du processus, permettant de le
        désigner, 
  \item état : actif, suspendu \ldots,
  \item commande exécutée par le processus.
\end{itemize}
Sans option, cela concerne les processus qui ont été lancés à partir du
terminal.

\begin{maw}
\begin{itemize}
  \item dans un terminal, tapez \cmd{ps},
  \item dans ce même terminal, tapez \cmd{evince \&},
  \item retapez \cmd{ps}.
\end{itemize}
\end{maw}

Cette commande \cmd{ps} dispose de nombreuses options parmi lesquelles on
trouve \cmd{ps -e} et \cmd{ps -f} (ou combinées \cmd{ps -ef}).

\begin{itemize}
  \item \cmd{ps -e} permet de lister l'ensemble des processus actifs de
        l'ordinateur,
  \item \cmd{ps -f} affiche les processus sous un format long.
\end{itemize}

Dans ce format long, quelques informations peuvent être utiles~:

\begin{itemize}
  \item le propriétaire du processus (a priori si vous êtes seul sur la
        machine (voir \ref{connexion}), il n'y a que le super-utilisateur
        \cmd{root} et vous),
  \item \cmd{PID} du processus père, c'est-à-dire du processus ayant lancé le
        processus considéré, et lui ayant ainsi transmis la plupart des
        éléments de son environnement d'exécution (variables,
        entrées/sorties\ldots). Tout processus a un père, sauf le processus
        initial, appelé \cmd{init}, de \cmd{PID} 1.
\end{itemize}
 
\begin{maw}
\begin{itemize}
  \item expérimentez les différentes options,
  \item comment lister l'ensemble des processus dont vous êtes propriétaire ?
        (deux solutions : utilisation de \cmd{grep} ou option supplémentaire)
\end{itemize}
\end{maw}

La commande \cmd{top}, disponible sur la plupart des \unix, fournit une liste
des processus analogue à celle donnée par la commande \cmd{ps}, où les
informations relatives à la consommation des ressources (mémoire vive, temps
processeur\ldots) sont rafraîchies et classées à intervalles réguliers.

\subsection{la commande \cmd{kill}}

Quelques fois, une application (par exemple graphique) refuse de répondre et en
particulier de s'arrêter, même en tuant la fenêtre (icône de la fenêtre avec
la croix).

Une façon de l'arrêter à coup sûr est de tuer le
processus correspondant. Pour cela, il faut utiliser la commande \cmd{kill} avec
comme paramètre l'identificateur du processus à tuer (le \cmd{PID}), d'où
l'utilité de connaître ce processus avec la commande \cmd{ps}.

\begin{maw}
\begin{itemize}
  \item dans un terminal, tapez \cmd{evince \&},
  \item retapez \cmd{ps -ef} et repérez le \cmd{PID} du processus
        \cmd{evince},
  \item tapez \cmd{kill -9 PID}, cette commande enverra le signal \cmd{KILL}
        au processus, ce qui le forcera à s'arrêter.
\end{itemize}
\end{maw}

\subsection{Processus en avant-plan et en arrière-plan}

Un processus peut être exécuté en arrière-plan. Dans ce cas, il s'éxécute en
parallèle avec le processus père, et n'interagit pas avec le terminal
(souvenez-vous du TP1 !).
En particulier, il est possible de lancer une commande en arrière plan à
partir du \shell, en la faisant suivre d'un \cmd{\&}. Ainsi~:

\begin{verbatim}
> evince &
>
\end{verbatim}

lance \cmd{evince}, et passe à l'attente de la ligne de commande suivante,
sans attendre la fin de la commande \cmd{evince}.

Il est possible, sur la plupart des \shell, de changer le mode d'exécution
d'un processus. Avec les \cmd{C-shell}, par exemple~:

\begin{itemize}
  \item la commande \cmd{jobs} permet de connaitre la liste des processus qui
        ne sont pas en avant-plan. Il faut noter que les identifiants de jobs
        de cette liste sont spécifiques au shell, et n'ont pas de rapport avec
        les \cmd{PID} fournis par la commande \cmd{ps}, 
  \item la commande \cmd{fg \%n} permet de poursuivre l'exécution du job
        \cmd{n} en avant-plan, 
  \item la commande \cmd{bg \%n} permet de poursuivre l'exécution du job
        \cmd{n} en arrière-plan.
\end{itemize}

\subsection{Suspension et reprise d'un processus}

Un processus en avant-plan peut être suspendu, c'est-à-dire que son exécution
est provisoirement arrêtée, mais pourra être reprise par la suite
(à partir de l'état atteint au moment de l'arrêt).

\begin{itemize}
  \item la frappe de \cmd{Ctrl-Z} permet de suspendre le processus en
        avant-plan, 
  \item les commandes \cmd{fg} ou \cmd{bg} permettent de reprendre l'exécution
        d'un processus suspendu, en avant-plan ou en arrière-plan. 
\end{itemize}

Vous verrez plus en détail les processus et les signaux au second semestre
dans le cours de \textsc{Systèmes Centralisés}.

\section{Droits d'accès et modification}

\subsection{Description des droits d'accès}

Nous avons vu que la commande \cmd{ls} permet de lister les
fichiers d'un répertoire. Cependant, cette commande possède
plusieurs modes (%\cmd{ls -h} ou 
\cmd{man ls}), notamment un mode
«~long~» qui permet d'afficher de nombreux détails sur les
fichiers : leurs droits d'accès, leur taille en octets, l'heure de dernière modification
et le nom. Il s'agit de l'option \cmd{-l} ({\em long}) :
\begin{verbatim}
> ls -l
total 4
drwxr-x---   2 camichel    in           512 Jul 29 09:54 CaML/
-rw-r-----   1 camichel    in           437 Jul 29 10:00 tolkien.txt
\end{verbatim}

Intéressons-nous à la suite de lettres et de tirets sur la
gauche : elle donne le type de fichier ainsi que ses droits
d'accès. Cette séquence de dix caractères est classiquement
séparée en quatre groupes :
\begin{center}
\begin{tabular}{cccc}
 \cmd{-} & \cmd{rw-} & \cmd{r--} & \cmd{r--} \\
 {\footnotesize Type de} & \multicolumn{3}{c}{\footnotesize Droits}\\
 {\footnotesize fichier} & {\footnotesize\tt u} & {\footnotesize\tt g} & {\footnotesize\tt o}\\
\end{tabular}
\end{center}

\subsubsection*{Type de fichier}

Le premier caractère renseigne le type de fichier :
\begin{itemize}
 \item \cmd{-} (tiret) : fichier normal (fichier texte, \ldots) ;
 \item \cmd{d} ({\em {\bf d}irectory}) : répertoire ;
 \item \cmd{l} ({\em {\bf l}ink}) : lien symbolique (\cmd{man ln}) ;
 \item etc\ldots (\cmd{man ls} pour en savoir plus).
\end{itemize}


\subsubsection*{Les droits}

Les neuf caractères restants sont séparés en trois groupes
de trois caractères~:
\begin{itemize}
 \item \cmd{u} : le premier groupe précise les droits du
                 propriétaire ({\em {\bf u}ser}) du fichier~;
 \item \cmd{g} : le deuxième groupe précise les droits des
                 autres utilisateurs du groupe\footnote{En
                 {\sc unix}, tout utilisateur fait partie d'au moins 
                 un groupe d'utilisateurs.}
                 ({\em {\bf g}roup}) du propriétaire~;
 \item \cmd{o} : le troisième groupe précise les droits des
                 autres utilisateurs extérieurs ({\em {\bf
                 o}thers}) au groupe.
\end{itemize}\noindent\underline{Remarque :} Il existe un super-utilisateur
(\cmd{root}), administrateur de l'ordinateur, qui possède tous les
droits et n'est donc pas concerné par ces restrictions.

\bigskip

Chaque groupe de trois caractères est construit de façon identique~:
\begin{itemize}
 \item le premier caractère précise s'il y a droit de
       lecture ({\em {\bf r}ead}) ou non, respectivement \cmd{r} ou
       \cmd{-} ;
 \item le deuxième caractère précise s'il y a droit
       d'écriture ({\em {\bf w}rite}) ou non, respectivement
       \cmd{w} ou \cmd{-} ;
 \item le troisième caractère précise s'il y a droit
       d'exécution\footnote{{\sc important :} Ce droit est
       indispensable pour pouvoir entrer dans un répertoire avec
       la commande \cmd{cd}.} ({\em e{\bf x}ecute}) ou non,
       respectivement \cmd{x} ou \cmd{-}.
\end{itemize}

\subsection{Modifications des droits : \cmd{chmod}}
\label{droits}
Pour modifier les droits des fichiers {\bf dont vous êtes le
propriétaire}\footnote{On rappelle que le propriétaire d'un fichier,
ainsi que son groupe,
est visible dans l'affichage long de \cmd{ls} (\cmd{ls -l}).}, il faut
utiliser la commande \cmd{chmod} dont la syntaxe est :

\smallskip

\hspace*{2cm} \cmd{chmod} [\cmd{-R}] \verb+<+{\em Droits}\verb+>+
\verb+<+{\em Fichiers}\verb+>+

\smallskip

L'argument {\em Droits} a deux écritures possibles :
\begin{itemize}
 \item la notation octale : chaque caractère de droit est
       représenté par un bit qui est à 1 si le droit est
       accordé et à 0 sinon. Chaque groupe est alors représenté
       \begin{itemize}
         \item soit par un nombre binaire sur trois bits, le caractère de
               droit en lecture étant le bit de poids fort,
         \item soit par la représentation de ce nombre en décimal (compris,
               donc, entre 0 et 7).
       \end{itemize}

       \noindent\underline{Exemple :}

       \begin{tabular}{ccccccccccc}
          \cmd{r}&\cmd{w}&\cmd{-}&~
         &\cmd{r}&\cmd{-}&\cmd{-}&~
         &\cmd{r}&\cmd{-}&\cmd{-}\\
          1&1&0&~
         &1&0&0&~
         &1&0&0\\
          \multicolumn{3}{c}{6} &~
         &\multicolumn{3}{c}{4} &~
         &\multicolumn{3}{c}{4} \\
        \end{tabular}
 \item la notation «~additive~» : au lieu de remplacer complètement
       les droits comme avec la notation octale, il est possible d'ajouter
       (ou d'enlever) certains droits localement. La notation utilisée
       est la suivante (principaux choix, faire \cmd{man chmod} pour
       plus de détails) :

       \hspace*{2cm}
       [\cmd{u}$|$\cmd{g}$|$\cmd{o}$|$\cmd{a}](\cmd{+}$|$\cmd{-}|\cmd{=})(\cmd{r}$|$\cmd{w}$|$\cmd{x})

       \begin{itemize}
         \item le premier caractère (optionnel) précise dans quel
               groupe d'uti\-li\-sa\-teurs la modification sera faite :
               s'il n'est pas précisé, la modification se fera pour tous
               (groupe \cmd{a}, {\em {\bf a}ll}) ; le caractère \cmd{a}
               indique que la modification est effectuée pour les trois
               groupes (\cmd{u}, \cmd{g} et \cmd{o})
         \item le deuxième caractère indique s'il faut ajouter
               (\cmd{+}), enlever (\cmd{-}) ou affecter (\cmd{=}) le droit ;
         \item le troisième caractère précise quel type de droit
               modifier : lecture (\cmd{r}), écriture (\cmd{w}) ou
               exécution (\cmd{x}). Le caractère \cmd{X} existe
               également et peut être très utile avec l'option
               \cmd{-R} de \cmd{chmod} (\cmd{man chmod} pour plus de
               détails).
       \end{itemize}
\end{itemize}
L'option \cmd{-R} (pour {\em {\bf R}ecursive}) permet d'appliquer
la modification de droits dans tous les fichiers (fichiers
classiques, répertoires, liens symboliques\ldots) de tous les
sous-répertoires des répertoires présents dans {\em
Fichiers}.

\begin{maw}[Travail]
Effectuez la séquence suivante :
\begin{verbatim}
> cd UtilisationDesOrdinateurs
> mkdir test
> chmod 700 test
> ls -l
> chmod -x test
> ls -l
> cd test
> chmod +x test
> ls -l
> cd test
\end{verbatim}
Que constatez-vous ? Comment l'expliquer ?
\end{maw}

\section{Connexion à un autre ordinateur}
\label{connexion}

\subsection{Indépendance du point de connexion}

Il est possible de se connecter à distance ({\em remote} en
anglais) à un autre ordinateur. %Dans le cadre des enseignements
%qui vous sont donnés à l'ENSEEIHT, une telle utilisation sera
%rare, mais il est bon de connaître cette fonctionnalité.

Le système de gestion des utilisateurs n'étant pas local,
vous pouvez vous connecter depuis votre ordinateur {\bf à n'importe
quel ordinateur du deuxième étage} : vous
retrouvez alors votre compte comme si vous étiez connecté
«~physiquement~» sur l'ordinateur (il n'y a en fait
quasiment aucune différence entre ces deux types de connexion)
et vous pouvez lancer des commandes normalement.

\subsubsection{Les commandes de connexion à distance}

Il existe plusieurs commandes pour effectuer cette connexion :
\begin{itemize}
 \item \cmd{telnet} : c'est un interpréteur de commandes en
       soi\footnote{Le \shell{} est un interpréteur de commandes.},
       mais il est souvent utilisé pour effectuer une connexion
       simple ;
 \item \cmd{ssh} ({\em {\bf S}ecure {\bf S}ession {\bf H}ost}) est
       l'une des plus utilisées.
\end{itemize}

\paragraph{Syntaxe}

Leur syntaxe de base est la suivante : \verb+<+{\em
commande}\verb+>+ \verb+<+{\em nomOrdinateur}\verb+>+ où {\em
commande} désigne l'une des quatre commandes. Le nom d'un ordinateur
est indiqué sur le dessus ou sur la façade de l'unité~centrale.

\subsubsection{Sécurité des commandes}

L'une des caractéristiques importantes de \cmd{ssh} 
est que \cmd{ssh} effectue des connexions
\underline{sécurisées}, c'est-à-dire que les transmissions
sont cryptées. De plus, \cmd{ssh} demande toujours
l'authentification, alors que ce n'est pas toujours le cas pour
les autres commandes.

\fbox{\centerline{\sc L'utilisation de \cmd{ssh} est fortement
encouragée.}}

\subsection{La commande \cmd{ssh}}

\subsubsection{Syntaxe}

Comme vu plus haut, la syntaxe de base est donc \cmd{ssh}
{\em nomOrdinateur}. Pour une syntaxe plus détaillée, se référer
aux pages \cmd{man} (\cmd{man ssh}).

\begin{maw}
Connectez-vous à un ordinateur quelconque.
\end{maw}

\subsubsection{Stockage de la clé cryptée de connexion}

Lors de votre première connexion par \cmd{ssh} à un ordinateur, vous
aurez le message du type suivant :
\begin{verbatim}
The authenticity of host '...' can't be established.
RSA key fingerprint in md5 is: xx:xx:xx:xx:xx:xx:xx:xx:xx:xx...
Are you sure you want to continue connecting(yes/no)?
\end{verbatim}
Tapez \cmd{yes} (en entier) et continuez.

Chaque machine est identifiée par une clé.
À la première connexion, on enregistre une empreinte de la clé. 

À chaque connexion ultérieure, on compare ce que la machine nous présente avec
ce que l'on a enregistré. Cela permet de détecter si une machine a été
remplacée par une autre.


\subsubsection{Déconnexion}

Pour finir la connection sécurisée et donc revenir sur votre ordinateur de
départ la commande est

\begin{verbatim}
exit
\end{verbatim}

\subsection{Autres types de connexion à distance}

À noter qu'il existe des possibilités de transfert de
fichiers avec :
\begin{itemize}
 \item \cmd{ftp} ({\em {\bf F}ile {\bf T}ransfer {\bf P}rotocol})
       et sa version sécurisée \cmd{sftp} ({\em {\bf S}ecure
       FTP}) qui fonctionnent comme des interpréteurs de
       commandes ;
 \item \cmd{scp} ({\em {\bf S}ecure {\bf C}o{\bf p}y}) qui
       a une syntaxe analogue (bien qu'un peu plus complexe) que
       \cmd{cp} et qui permet d'effectuer des copies de façon
       directe.\\ \underline{N.B. :} \cmd{scp} utilise \cmd{ssh}.
\end{itemize}

% \section{Savoir qui est là\ldots}

\subsection{Qui est sur «~mon~» ordinateur ?}

Un point important : plusieurs utilisateurs peuvent être
connectés {\bf sur le même ordinateur en même temps} (personne
assise devant l'ordinateur, \cmd{ssh}\ldots).

Une fois connecté sur un ordinateur, vous avez alors la
possibilité de voir qui est connecté sur cet ordinateur : pour
cela vous disposez de trois commandes~: \cmd{w}, \cmd{who} et
\cmd{finger} (sans aucun paramètre).

\begin{maw}
 Connectez-vous sur un ordinateur quelconque et essayez ces trois
 commandes.
\end{maw}

% \subsection{Y a-t-il quelqu'un ?}
% 
% Vous pouvez également avoir une vue d'ensemble du réseau en
% utilisant la commande \cmd{rusers}. Si vous voyez un même login
% plusieurs fois, cela signifie que cet utilisateur a plusieurs
% connexions ($\simeq$ terminaux en général) sur cet ordinateur. Habituellement
% la commande \cmd{rusers} met du temps à se terminer ; si vous voulez
% l'arrêter, tapez \cmd{Ctrl-C}.

% \subsection{La commande \cmd{finger}}
% 
% Nous venons de voir la commande \cmd{finger} sans paramètre. En
% réalité, cette commande permet de faire la correspondance
% entre login et nom réel (\cmd{man finger} pour la syntaxe).
% 
% \begin{maw}
% 
% \
% 
%  \begin{Question}
%   Cherchez le nom réel d'un login de votre choix.
%  \end{Question}
%  \begin{Question}
%   Cherchez tous les logins dont le nom ou le prénom contiennent
%   votre prénom.
%  \end{Question}
%  \begin{Question}
%   En utilisant \cmd{rusers} et \cmd{finger}, déterminez les
%   personnes réelles connectées sur trois ordinateurs de votre
%   choix.
%  \end{Question}
% \end{maw}


\subsection{L'affichage à distance d'applications graphiques}

Lorsque vous vous connectez sur un ordinateur distant, vous pouvez
lancer n'importe quelle application comme si vous étiez sur
votre propre ordinateur\ldots à un détail près : l'affichage
graphique.

Par défaut, l'affichage graphique se fait sur l'ordinateur depuis
lequelle vous avez appelé l'application, qui ---~si vous vous êtes
connecté à distance~--- n'est pas la vôtre. Autrement dit,
l'application aura bien été lancée mais vous ne verrez rien !%
\footnote{Nous parlons ici d'affichage de fenêtres graphiques.
L'affichage dit «~en mode texte~» (dans le terminal) n'est pas
affecté.}

\begin{comment}
Les flux de données graphiques ne sont pas, par défaut,
automatiquement transmis à travers la connexion à l'ordinateur
distante. Il faut explicitement l'autoriser.
\end{comment}

\paragraph{Autoriser le flux de données graphiques avec \cmd{ssh}}

\

Dans le cas où l'option n'est pas activée, l'autorisation de flux graphique
se fait en ajoutant l'option~\cmd{-X} à la commande \cmd{ssh}~:

\noindent \verb+> ssh -X+ {\em nomOrdinateur}

% Il s'avère que sur les ordinateurs de l'ENSEEIHT, cette option est déjà activée.
% Avec un \cmd{ssh} simple, le flux est redirigé automatiquement vers
% l'ordinateur source. Attention, généralement l'option n'est pas activée.
% 
\begin{maw}
 Exécutez \cmd{evince} sur un autre ordinateur que le vôtre.
\end{maw}


\section{Les variables d'environnement}

Votre environnement de travail est en partie défini par un
certain nombre de variables utilisées par le \shell ; ce
sont les variables d'environnement. Nous allons en étudier
quelques-unes.

\subsection{Quelques variables d'environnement}

Quelques petits points :
\begin{itemize}
 \item les variables d'environnement sont conventionnellement en
       \cmd{MAJUSCULES} ;
 \item le \cmd{\$} qui précède la variable signifie que l'on
       en prend la valeur.
\end{itemize}

Voici quelques variables parmi les plus importantes :
\begin{itemize}
 \item \cmd{HOME}    : cette variable contient le chemin absolu de
                       votre répertoire de connexion et ne doit pas être
                       modifiée ;
 \item \cmd{PATH}    : cette variable contient une liste de répertoires dans
                       lesquels chercher les fichiers désignés par des chemins
                       relatifs (dont les commandes vues jusqu'à présent) ;
% \item \cmd{DISPLAY} : cette variable contient l'adresse de
%                       l'endroit où effectuer les affichages
%                       graphiques.
\end{itemize}
Ainsi, \cmd{cd} (appel sans paramètre) est équivalent à
\cmd{cd \$HOME}.

\subsection{Manipuler les variables d'environnement}

Vous avez à votre disposition
\begin{itemize}
\item \cmd{echo} qui permet afficher une chaîne et en particulier les
variables d'environnement :
\begin{verbatim}
> echo $PATH
> echo $HOME
\end{verbatim}

\smallskip
\
Noter la présence du \cmd{\$}. (Essayez \cmd{echo PATH} pour
voir la différence)
\item l'affectation \cmd{=} qui permet de modifier une variable :
\begin{verbatim}
> PATH=${PATH}:/usr/local/bin/
\end{verbatim}
\item \cmd{export} qui permet de rendre visibles les valeurs des variables
d'environnement à des shells ou commandes consécutifs au shell courant :
\begin{verbatim}
> export PATH
\end{verbatim}
\item souvent l'affectation et l'exportation se font en une seule commande :
\begin{verbatim}
> export PATH=${PATH}:/usr/local/bin/
\end{verbatim}
\end{itemize}

% Par contre, il est déconseillé de modifier la variable \cmd{HOME}.

\subsection{Retour sur \cmd{PATH}}

\subsubsection{La commande \cmd{which}}
Une commande intéressante est la commande \cmd{which} qui permet
de retrouver l'endroit où se situe un exécutable {\bf à
condition que le chemin cherché fasse partie des chemins
spécifiés dans \cmd{PATH}} (c'est-à-dire si vous pouvez
l'exécuter sans donner le chemin complet).

\subsubsection{\cmd{source}}

\cmd{source} {\em Fichier} exécute toutes les commandes
\shell{} incluses dans le fichier {\em Fichier}.

\begin{maw}

\begin{itemize}
 \item rechercher le chemin de \cmd{chmod}, \cmd{matlab}\ldots et
       \cmd{which},
 \item effectuez la séquence suivante :
\begin{verbatim}
> cd UtilisationDesOrdinateurs
> mkdir test
> cp ~guivarch/pub/helloworld ~/UtilisationDesOrdinateurs/test
> helloworld
> ./test/helloworld
> export PATH=${HOME}/UtilisationDesOrdinateurs/test:$PATH
> echo $PATH
> helloworld
> which helloworld
\end{verbatim}
Que constatez-vous ? Comment l'expliquer ?
\end{itemize}
\end{maw}

\paragraph{Dernières remarques :}
\
\begin{itemize}
 \item la commande \cmd{mkdir test} peut indiquer que le répertoire \cmd{test}
       existe déjà si vous avez suivi correctement les instructions de
       l'exercice de la section \ref{droits},
 \item le raccourci \verb+~+ est équivalent à votre répertoire de connexion~;
       du coup, \verb+~+ permet de donner facilement un chemin relatif à
       un répertoire de connexion.
 \item la commande \cmd{cp \~{}guivarch/pub/helloworld
 \~{}/UtilisationDesOrdinateurs/test} veut dire :\\
       copier le fichier \cmd{helloworld} qui est sur le compte \cmd{guivarch}
       dans le répertoire \cmd{pub} vers le répertoire\\
       \cmd{UtilisationDesOrdinateurs/test} de votre compte.
\end{itemize}

\section{CQFAR (Ce Qu'il Faut Avoir Retenu)}

\begin{itemize}
 \item savoir identifier les processus en cours d'exécution,
 \item savoir tuer le processus associé à une application qui ne répond plus,
 \item savoir déchiffrer les droits d'un fichier \unix,
 \item savoir modifier les droits d'un fichier \unix,
 \item savoir se connecter à distance sur un autre ordinateur (\cmd{ssh}),
 \item savoir afficher une application graphique lancée à distance depuis
       un autre ordinateur,
 \item connaître les variables d'environnement \cmd{HOME} et
       \cmd{PATH} ; savoir modifier cette dernière.
\end{itemize}

\end{document}
