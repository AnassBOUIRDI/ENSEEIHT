%%%%%%%%%%%%%%%%%%%%%%%%%%%%%%%%%%%%%%%%%%%%%%%%%%%%%%%%%%%%%%%%%%%%%%%%%%%%%%%%
%           Auteur :  Patrick Sallé
%         Création :  01/08/04              
%         Modifications 
%              (Ph. Mauran, 21/09/07) : shell, tubes, redirections; compléments 
%              sur les fichiers et les variables d'environnement
%%%%%%%%%%%%%%%%%%%%%%%%%%%%%%%%%%%%%%%%%%%%%%%%%%%%%%%%%%%%%%%%%%%%%%%%%%%%%%%%
\documentclass[a4paper,11pt]{article}

\usepackage{../tp_utf8}

%====[ PAGE DE TITRE ]==========================================================
\title{Coquillages et crustacés : quelques commandes \textsc{Shell}}
\numero{TP n°3}
\date{2013--2014}
\discipline{Utilisation des Ordinateurs}
\promotion{ENSEEIHT --- 1\iere Année Informatique}

\newcommand{\unix}{\textsc{unix}}
\newcommand{\linux}{\textsc{linux}}
\newcommand{\shell}{\textsc{shell}}


%====[ DEBUT DU DOCUMENT ]======================================================

\begin{document}

\maketitle

\section*{Objectifs}
\begin{itemize}
  \item Interpréteur de commandes :  \shell,
  \item Archivage et compression de données,
  \item Les commandes de gestion des répertoires et des fichiers,
  \item Redirections et tubes,
  \item La commande alias,
  \item Les variables d'environnement,
  \item Les fichiers de configuration du shell.
\end{itemize}

\

\begin{danger}
  Le sujet étant assez dense et certaines notions plus avancées, nous avons
  distingué 2 niveaux de difficultés et/ou d'utilité ;
  nous souhaitons que vous abordiez toutes les notions de niveaux $*$.
\end{danger}

\section{Interpréteur de commandes :  \shell \; (niveau $*$)}

Lorsqu'un utilisateur se connecte, un interpréteur de commandes (ou
\shell{} dans la terminologie \unix/\linux)
est lancé\footnote{dans la suite du document, nous utiliserons le terme
\unix, \linux{} étant un type d'\unix}.
Cet interpréteur a pour charge de recueillir les commandes de l'utilisateur
et lancer leur exécution.
Le  \shell{} exécute donc une boucle sans fin dont chaque pas consiste
à :
\begin{itemize}
  \item lire une ligne de commande (terminée par une frappe sur la touche
        \textsc{Entrée}),
  \item interpréter la ligne saisie : le \shell{} propose en effet un certain
        nombre de raccourcis (jokers, alias\ldots (voir plus loin)), destinés
        à faciliter la saisie des commandes. L'interprétation de la ligne va
        donc consister à remplacer les abréviations par la forme complète
        correspondante,
  \item lancer l'exécution de la commande.
\end{itemize}

Une ligne de commande se présente comme une suite de mots, séparés par des
espaces. La forme générale d'une ligne de commande est :

 \verb!chemin -options paramètres!
 
où $chemin$, qui est le premier mot, est le chemin d'accès au fichier
exécutable correspondant à la commande. Les options se distinguent
(généralement) des paramètres en ce qu'elles sont précédées d'un tiret.
Toutes les options et tous les paramètres sont séparés par des espaces.

Il existe deux grandes familles de \shell{} : \textsc{Bourne-Shell} \cmd{sh} et
\textsc{C-Shell} \cmd{csh} et leurs dérivés (\cmd{bash}, \cmd{tcsh}, ...).
C'est le \shell{} \cmd{bash} qui vous utiliserez cette année.

Nous allons manipuler dans cette séance quelques commandes utiles du
\shell{}.

\section{Archivage (niveau $*$)}

Il est possible en \unix{} d'archiver plusieurs fichiers ou répertoires
en les regroupant dans un fichier archive unique.
Dans le cas d'un répertoire toute l'arborescence de fichiers dont ce
répertoire est la racine est archivée.
Cet archivage est utile pour des sauvegardes ou pour communiquer facilement un
ensemble de fichiers.

Pour effectuer l'archivage et le désarchivage, on utilise la commande
\cmd{tar} :

\begin{itemize}
  \item Archivage : \cmd{tar -cvf arch.tar file1 file2 rep1} \\
        regroupe dans le fichier arch.tar les deux fichiers file1 et file2
        ainsi que le contenu du répertoire rep1, en préservant
        l'arborescence.
\end{itemize}

\begin{danger}
\textbf{Bien spécifier en premier paramètre un nom de fichier d'archivage avec
son suffixe .tar, sinon c'est le nom du premier fichier que vous voulez
archiver qui sera utilisé comme nom d'archive et donc votre premier fichier
sera écrasé et perdu.}
\end{danger}

\begin{itemize}
  \item Désarchivage : \cmd{tar -xvf arch.tar} extrait le contenu du fichier
        archive,
  \item Consultation : \cmd{tar -tvf arch.tar} liste le contenu du
        fichier archive sans extraire.
\end{itemize}

Faire \cmd{man tar} pour plus d'informations.

\

\begin{maw}

\begin{itemize}
  \item Créer un répertoire \cmd{TP3} dans votre répertoire
        \cmd{UtilisationDesOrdinateurs},
  \item Récupérer sur la page Utilisation des Ordinateurs,
        le fichier sujets.tar,
  \item Dés-archiver cette archive et parcourez-la pour découvrir son
        contenu,
  \item Détruire le fichier sujets.tar,
  \item Reconstruire un fichier d'archive identique à l'original.
\end{itemize}
\end{maw}

\begin{danger}[Conseil]
\textbf{pour éviter de mal utiliser la commande \cmd{tar} et
d'écraser par mégarde des fichiers, nous vous conseillons d'archiver des
répertoires. Par exemple si vous voulez archiver le répertoire TP3, il
suffira de  :
\begin{itemize}
  \item se placer dans le répertoire parent de TP3
  \item taper \cmd{tar -cvf TP3.tar TP3}
\end{itemize}
Un oubli de \cmd{TP3.tar} entraînera une erreur d'utilisation de \cmd{tar}
mais en aucun cas ne détruira de fichier.}
\end{danger}

\begin{danger}[Conseil]
Pensez à effacer les fichiers archive une fois que vous avez dé-archivé leur
contenu. Dans beaucoup d'enseignements, vous allez récupérer de tels fichiers
et, à la longue, vous risquez de ne plus avoir de place mémoire (quota).
\end{danger}

\section{Gestion des répertoires et des fichiers (niveau $*$)}

On rappelle que tout fichier ou répertoire peut être désigné :
\begin{itemize}
  \item soit par un chemin absolu depuis la racine / 

        \verb!/bin/ls! est le chemin d'accès au programme qui exécute la
        commande ls,
  \item soit par un chemin relatif par rapport au répertoire de travail
        (\emph{working directory} affiché par la commande \cmd{pwd}), aussi
        appellé répertoire courant.
        Initialement (en début de session \shell), le répertoire travail est le
        répertoire de connexion.  La commande \verb!cd! permet de 
        changer le répertoire de travail (voir TP1) :

        \verb!UtilisationDesOrdinateurs/TP3! est un chemin d'accès à partir de votre
        répertoire de travail.
\end{itemize}


\subsection{Copie, Déplacement, Suppression, Renommage (niveau $*$)}

Les commandes \cmd{cp} et \cmd{mv} utilisent un «argument source» et un
«argument destination». Ces arguments sont un chemin d'accès relatif
ou absolu vers un fichier ou un répertoire. L'action effectuée dépend
de la nature de la source et de la destination :

\begin{itemize}
  \item \cmd{cp f1 f2} : fait une copie du fichier f1 dans un fichier f2
        dont le chemin d'accès sera f2. 
        Si un fichier ayant ce chemin d'accès existe, il est écrasé par la
        copie de f1,\\
  \item \cmd{cp f1 ... fn rep} : fait une copie des fichiers f1, \ldots,
        fn dans le répertoire rep (existant) en préservant le nom des
        fichiers,\\
  \item \cmd{mv f1 f2} : déplace le fichier f1  dans un fichier dont le chemin
        d'accès sera f2. Si un fichier ayant ce chemin d'accès existe il est
        écrasé par f1, \\
  \item \cmd{mv f1 ... fn rep} : déplace les fichiers f1 \ldots fn dans le
        répertoire existant  dont le chemin d'accès est rep.
        Les fichiers de même nom dans ce répertoire seront écrasés,\\
  \item \cmd{mv rep1 rep2} : 
        \begin{itemize}
          \item si le répertoire désigné par le chemin d'accès rep2 existe
                alors le répertoire rep1 et son contenu est transféré dans le
                répertoire rep2 dont il devient un sous-répertoire,
          \item s'il n'existe pas, le répertoire rep1 est renommé en rep2
                (changement du chemin d'accès).
        \end{itemize}
\end{itemize}

Les commandes \cmd{rm} et \cmd{rmdir} suppriment respectivement des
fichiers et des répertoires vides, désignés par leur chemin d'accès :

\begin{itemize}
  \item \cmd{rm f1 ... fn} : supprime les fichiers f1 \ldots fn,
  \item \cmd{rmdir rep} : supprime le répertoire rep si celui-ci est vide.
\end{itemize}

% Les trois commandes \cmd{cp}, \cmd{mv} et \cmd{rm} que vous avez à votre
% disposition sont des \cmd{alias} (voir section \ref{alias}) et l'option
% \cmd{-i} (interactif) est activée. Les commandes demandent donc
% confirmation :
% 
% \begin{itemize}
%   \item pour \cmd{cp} et \cmd{mv} quand la destination est un fichier existant,
%   \item dans tous les cas pour le \cmd{rm}.
% \end{itemize}

\subsection{Caractères joker (niveau $*$)}
Lorsque le  \shell{} interprète les noms de fichiers, il considère certains
caractères comme des caractères spéciaux (jokers, aussi appelés
\emph{métacaractères})~:

\begin{itemize}
  \item \verb|*| correspond à un nombre quelconque de caractères,
  \item \verb|?| correspond  à exactement un caractère,
  \item \verb|[abcdgh]| correspond  à exactement un caractère de
        l'ensemble {a,b,c,d,g,h},
  \item \verb|\| indique que le caractère qui suit ne doit pas être considéré
        comme un métacaractère, ou un caractère spécial du \shell ($<$, $>$,
        \verb+|+) ; cela permet, par exemple de désigner des fichiers dont le
        nom comporte des métacaractères.
 \end{itemize}

Par exemple~:

\begin{itemize}
  \item \cmd{more *.ml} : fait un \cmd{more} sur tous les fichiers
        dont le nom est terminé par .ml,
\\  
  \item \cmd{ls *.*} : liste tous les fichiers du répertoire courant dont le
        nom contient un point,
\\  
  \item \cmd{rm a/*} : supprime tous les fichiers du répertoire \verb+a+ du
        répertoire courant dont le nom ne commence pas
        par un point (i.e. tous les fichiers du répertoire \verb+a+ 
        sauf les fichiers cachés),
\\  
  \item \cmd{ls TP?/*.c} : liste tous les fichiers dont le nom a pour suffixe
        \cmd{.c}, et qui sont en outre contenus dans des répertoires dont le nom
        est composé de trois lettres commence par \cmd{TP},
\\  
  \item \cmd{ls *?.*?.*} : liste tous les fichiers dont le nom contient au
        moins un caractère puis un point puis au moins un caractère puis un
        point puis un nombre quelconque (nul éventuellement) de caractères.
\end{itemize}

\

\begin{maw}

L'archive que vous avez extraite contient les premiers sujets de
TP. On vous demande de réorganiser cet ensemble de répertoires en :

\begin{itemize}
  \item créant deux nouveaux répertoires «Sources» et «PDF»,
  \item transférant dans Sources tous les fichiers .tex et dans PDF tous
        les fichiers .pdf,
  \item supprimant les anciens répertoires et les fichiers qui subsistent,
  \item reconstituant une archive contenant la nouvelle structure.
\end{itemize}

Vous pouvez recommencer plusieurs fois en utilisant des commandes
différentes et en utilisant des chemins relatifs et absolus.
\end{maw}

\section{Redirections (niveau $**$)}

Les calculs en cours (ou processus) échangent des informations avec leur
environnement (périphériques, fichiers, ...) sous la forme de suites d'octets
(ou flots d'entrées/sorties).

Initialement, le \shell{} dispose de trois flots d'entrée/sortie :
\begin{itemize}
  \item l'entrée standard (flot numéro $0$), associée au clavier,
  \item la sortie standard (flot numéro $1$), associée à l'écran,
  \item la sortie erreur standard (flot numéro $2$), associée aussi à l'écran.
\end{itemize}
Ces associations des flots standard sont conservées, par défaut pour les
commandes lancées à partir du \shell. Néanmoins, il est possible de redéfinir
ces associations, pour une commande donnée : on parle alors de redirection
d'un flot. Ainsi :
\begin{itemize}
  \item la sortie standard peut être définie à l'aide du caractère \verb!>! :\\
        \cmd{ls -l > liste} : permet de récupérer le résultat de  \verb!ls!
        dans le fichier $liste$,
\\
  \item il est possible de conserver le contenu d'un fichier vers lequel on
        redirige la sortie standard :\\
        \cmd{ls -l >> liste} : ajoute le résultat de \verb!ls!  a la fin du
        fichier $liste$,
\\
  \item en Bourne Shell  (\verb!sh! ou \verb!bash!), la sortie erreur standard
        peut être redirigée par  \verb!2>!\\
        \cmd{rm * 2> log} : écrit les messages d'erreurs éventuellement
        engendrés par l'exécution de  \verb!rm!  dans le fichier $log$,
\\
  \item l'entrée standard peut être redirigée par  \verb!<!\\
        \cmd{sort < tolkien.txt} classe les lignes du fichier tolkien.txt par
        ordre alphabétique.

\end{itemize}

\section{Tubes (niveau $**$)}

La sortie standard d'une commande peut être reliée à (redirigée vers) l'entrée
standard d'une autre commande par un $tube$ (ou $pipe$ en anglais).
La seconde commande peut alors utiliser comme  point de départ de son
traitement les résultats fournis par la première commande. Ce schéma de
coordination est appelé un schéma $producteur/consommateur$ et sera revu
longuement lorsque vous suivrez les cours de \textsc{Systèmes Centralisés}.

L'utilisation d'un tube permet de transmettre un flot de données entre deux
commandes sans avoir à passer par un fichier intermédiaire : les données
transmises peuvent être conservées dans un tampon, en mémoire vive.

La mise en place d'un tube est définie à l'aide du caractère \verb!|! obtenu
en appuyant simultanément sur les touches «Alt Gr» et «-»/«6».

\cmd{ls -l | more} : relie la sortie de  \verb!ls -l! à l'entrée de
\verb!more!. La liste des fichiers du répertoire courant sera traitée par
\verb!more!, c'est-à-dire affichée page à page (bien utile quand le nombre de
fichiers contenus dans un répertoire est important).

\section{Compression (niveau $*$)}
Lorsqu'une archive (ou un fichier quelconque) est trop volumineuse, on
peut la compresser.

\begin{itemize}
  \item \cmd{gzip rep.tar} : remplace rep.tar par rep.tar.gz compressé,
  \item \cmd{gunzip rep.tar.gz} réalise l'opération inverse.
\end{itemize}

Si vous désirez produire ou récupérer des archives du monde
\textsc{Windows} vous pouvez utiliser les commandes :
\begin{itemize}
  \item \cmd{zip -r9 rep.zip rep} : qui crée l'archive compressée rep.zip à
        partir du répertoire rep,
  \item \cmd{unzip rep.zip} qui réalise l'opération inverse.
\end{itemize}

\

L'option \cmd{-r9} sélectionne une compression maximale ; vous pourrez tester
l'option \cmd{-r1} pour voir la différence.

\begin{maw}

\begin{itemize}
  \item compresser et décompressez l'archive produite précédemment,
  \item comparer les tailles avec \cmd{ls -l}
\end{itemize}

\end{maw}

\section{Commandes récursives (niveau $**$)}

Les principales commandes de manipulation de l'arborescence des
fichiers possèdent une option récursive \verb!-R! qui s'appliquent à un
répertoire et à tous ses sous-répertoires :

\begin{itemize}

  \item \cmd{ls -R} : liste récursivement dans un parcours «en
        largeur» un répertoire et ses sous-répertoires,\\
  \item \cmd{cp -R rep1 rep2} : fait une copie récursive du répertoire
        rep1
        \begin{itemize}
            \item  dans un nouveau répertoire de chemin d'accès rep2, si rep2
                   n'existe pas,
            \item dans un sous-répertoire de nom rep1, dans le répertoire rep2
                  s'il existe déjà!\\
          \end{itemize}
  \item \cmd{rm -R rep} : détruit tout le répertoire rep et son contenu
        récursivement! Utiliser avec prudence ou avec l'option \verb!-i!.
\end{itemize}

\

\begin{maw}

Expérimenter les options récursives sur l'archive.

\end{maw}

\begin{comment}
\begin{danger}
Pour la suite du sujet, tapez la commande suivante dans un terminal
\begin{verbatim}
echo $SHELL
\end{verbatim}
Si le résultat est \verb+/bin/tcsh+, les sections \ref{alias_tcsh}, \ref{tcsh}, \ref{cshrc} et
\ref{grep} terminent le sujet.

Si le résultat est \verb+/bin/bash+, les sections \ref{alias_bash}, \ref{bash}, \ref{bashrc} et
\ref{grep} terminent le sujet.
 
\end{danger}
\end{comment}

\begin{comment}
\section{Alias (cas de \cmd{tcsh}) (niveau $*$)}
\label{alias_tcsh}

La commande \texttt{alias} permet de d'abréger l'écriture
en renommant une fonction ou une expression plus complexe. Par
exemple :
\begin{verbatim}
alias ll 'ls -l'
alias del 'rm -i'
\end{verbatim} 

Le deuxième \texttt{alias} permet d'utiliser systématiquement la
version sécurisée de la commande \texttt{rm} qui demande confirmation
avant suppression de chaque fichier concerné.

La liste des alias définis est consultable en tapant simplement la
commande \texttt{alias} sans argument :
\begin{verbatim}
alias
\end{verbatim}

Vous pouvez définir un alias sur une commande en utilisant le nom initial de
la commande ; cela modifiera l'effet de la commande~:

\begin{verbatim}
alias mv 'mv -i'
alias cp 'cp -i'
\end{verbatim} 

Avec ces alias, lorsque vous déplacerez ou copierez des fichiers/répertoires,
il y aura une demande de confirmation en cas de conflit ce qui évitera
quelques pertes.

Pour pouvoir utiliser la commande initiale, sans l'effet de l'alias, vous
devrez la faire précéder d'un \verb+\+.

\begin{verbatim}
\mv fic1 fic2
\end{verbatim}

\begin{maw}
Définir des alias et expérimenter leur effet.
\end{maw}

\paragraph{Remarque :} la portée d'un alias, c'est à dire l'environnement dans
lequel l'alias est défini, est le \shell{} courant autrement dit, en
simplifiant, le terminal courant.
Vous pouvez le constater en manipulant deux terminaux et
en vérifiant que les alias de l'un ne sont pas connus de l'autre (à moins
bien sûr, de définir les mêmes alias dans les deux terminaux).

On verra à la section \ref{cshrc} comment rendre les alias « permanents ». 

\section{Variables d'environnement (cas de \cmd{tcsh})}
\label{tcsh}

L'interprète de commandes permet de définir des variables d'environnement,
c'est-à-dire d'associer des noms à des chaînes de caractères. Cette facilité
permet de configurer simplement le comportement de l'interprète de commandes
en associant des valeurs particulières à des variables prédéfinies. Par
exemple, la variable d'environnement \cmd{history} indique le nombre maximum de
commandes tapées dans le \shell{} qui sont mémorisées et accessibles par
les flèches haut/bas.

Au cours de ce TP nous étudierons les variables locales dont la portée
est l'interprète \shell{} dans lequel elles ont été définies.

\begin{itemize}
  \item \cmd{set} liste les variables locales et leur valeur,
  \item \cmd{echo \$$variable$} : affiche la valeur d'une  variable,
  \item \cmd{set variable = chaîne} :  modifie la valeur d'une variable
        locale (ou en crée une  nouvelle). 
\end{itemize}

\begin{maw}

\begin{itemize}
  \item utiliser la commande \cmd{set}, puis la commande \cmd{echo} pour
        consulter les variables déjà définies,
  \item modifier la variable \cmd{history} en lui donnant la valeur 2 et
        vérifier que le nombre de commandes mémorisées est 2,
  \item modifier après l'avoir consultée la variable \texttt{prompt} qui
        définit l'«invite» de l'interpréteur \shell{} :
        \begin{itemize}
          \item  essayez \texttt{set prompt =    "Bonjour :"} ,
          \item regardez l'action de \%Cn et \%T dans la chaîne ($n$ désigne
                un entier), 
        \end{itemize}
  \item constater que cette modification n'a pas d'influence sur d'autres
        occurences de \shell{} s'exécutant dans d'autres fenêtres.
%        \footnote{les variables à portée globale seront abordées au TP suivant}
\end{itemize}

\end{maw}

\section{Fichiers d'initialisation de \cmd{tcsh}}
\label{cshrc}

Le répertoire de connexion de votre compte contient des fichiers
« cachés », dont le nom commence par un point. Ils sont listés par la
commande

\verb+    ls -a+

Lors de la connexion d'un utilisateur, l'interprète de commandes
exécute les instructions contenues dans le fichier \cmd{.login}. Ces
instructions ne sont donc exécutées qu'une seule fois.

Les commandes  du fichier \cmd{.cshrc} sont exécutées chaque fois qu'un
interprète \textsc{C-shell} est lancé.

Lors de la déconnexion c'est le fichier \cmd{.logout} qui est pris en compte.

Dans ces fichiers, on peut :

\begin{itemize}
  \item définir des alias,
  \item modifier des chemins d'accès,
  \item configurer des programmes,
  \item \ldots
\end{itemize}

Pour prendre en compte les modifications effectuées dans les fichiers de
configuration, vous pouvez :

\begin{itemize}
  \item soit vous déconnecter puis vous reconnecter sur la machine, dans le
        cas des fichiers \cmd{.logout} et \cmd{.login},
  \item soit lancer un nouveau \textsc{C-shell} (commande \cmd{tcsh}),
  \item soit utiliser la commande \cmd{source}  qui exécute les instructions
        contenues dans un fichier (\cmd{source .login} ou \cmd{source .cshrc}).
\end{itemize}

\

\begin{maw}
\begin{itemize}
  \item éditer le fichier \cmd{.cshrc} pour introduire des alias ou
        modifier la variable prompt,
  \item exécuter la commande \cmd{source .cshrc} et constater les
        changements,
  \item faire la même constatation dans tout nouveau terminal exécutant
        un \shell.
\end{itemize}
\end{maw}

\newpage
\end{comment}

\section{Alias %(cas de \cmd{bash})
(niveau $*$)}
\label{alias_bash}

La commande \texttt{alias} permet de d'abréger l'écriture
en renommant une fonction ou une expression plus complexe. Par
exemple :
\begin{verbatim}
alias ll="ls -l"
alias del="rm -i"
\end{verbatim} 

Le deuxième \texttt{alias} permet d'utiliser systématiquement la
version sécurisée de la commande \texttt{rm} qui demande confirmation
avant suppression de chaque fichier concerné.

La liste des alias définis est consultable en tapant simplement la
commande \texttt{alias} sans argument :
\begin{verbatim}
alias
\end{verbatim}

Vous pouvez définir un alias sur une commande en utilisant le nom initial de
la commande ; cela modifiera l'effet de la commande~:

\begin{verbatim}
alias mv="mv -i"
alias cp="cp -i"
\end{verbatim} 

Avec ces alias, lorsque vous déplacerez ou copierez des fichiers/répertoires,
il y aura une demande de confirmation en cas de conflit ce qui évitera
quelques pertes.

Pour pouvoir utiliser la commande initiale, sans l'effet de l'alias, vous
devrez la faire précéder d'un \verb+\+.

\begin{verbatim}
\mv fic1 fic2
\end{verbatim}

\begin{maw}
Définir des alias et expérimenter leur effet.
\end{maw}

\paragraph{Remarque :} la portée d'un alias, c'est à dire l'environnement dans
lequel l'alias est défini, est le \shell{} courant autrement dit, en
simplifiant, le terminal courant.
Vous pouvez le constater en manipulant deux terminaux et
en vérifiant que les alias de l'un ne sont pas connus de l'autre (à moins
bien sûr, de définir les mêmes alias dans les deux terminaux).

On verra à la section \ref{bashrc} comment rendre les alias « permanents ». 


\section{Variables d'environnement %(cas de \cmd{bash})
(niveau $*$)}
\label{bash}
L'interprète de commandes permet de définir des variables d'environnement,
c'est-à-dire d'associer des noms à des chaînes de caractères. Cette facilité
permet de configurer simplement le comportement de l'interprète de commandes
en associant des valeurs particulières à des variables prédéfinies. Par
exemple, la variable d'environnement \cmd{PWD} contient la valeur du
répertoire courant.

Au cours de ce TP nous étudierons les variables locales dont la portée
est l'interprète \shell{} dans lequel elles ont été définies.

\begin{itemize}
  \item \cmd{env} liste les variables locales et leur valeur,
  \item \cmd{echo \$$variable$} : affiche la valeur d'une  variable,
  \item \cmd{variable=chaîne} :  modifie la valeur d'une variable
        locale (ou en crée une  nouvelle). 
\end{itemize}

\begin{maw}

\begin{itemize}
  \item utiliser la commande \cmd{env}, puis la commande \cmd{echo} pour
        consulter les variables déjà définies,
  \item consulter la valeur de la variable \cmd{HISTSIZE}, qui indique le
        nombre maximum de commandes tapées dans le \shell{} mémorisées et
        accessibles par les flèches haut/bas 
  \item donner lui  la valeur 2 et vérifier que le nombre de commandes
        mémorisées est 2,
  \item modifier après l'avoir consultée la variable \texttt{PS1} (prompt) qui
        définit l'«invite» de l'interpréteur \shell{} :
        \begin{itemize}
          \item essayez \texttt{export PS1="Bonjour :"} ,
          \item regardez l'action de \verb+\u+, \verb+\w+, \verb+\W+ et
                \verb+\!+ dans la chaîne (en changeant de répertoire pour
                \verb+\w+ et \verb+\W+),
        \end{itemize}
  \item constater que cette modification n'a pas d'influence sur d'autres
        occurences de \shell{} s'exécutant dans d'autres fenêtres.
%        \footnote{les variables à portée globale seront abordées au TP suivant}
\end{itemize}

\end{maw}

\section{Fichiers d'initialisation de \cmd{bash} (niveau $*$)}
\label{bashrc}

Le répertoire de connexion de votre compte contient des fichiers
d'initialisation « cachés », dont le nom commence par un point :
\cmd{.bashrc}, \cmd{.bash\_login}, \cmd{.bash\_logout}.

Ils sont listés par la commande

\verb+    ls -a+

\textbf{Ils n'existent pas obligatoirement par défaut.}

Lors de la connexion d'un utilisateur, l'interprète de commandes
exécute les instructions contenues dans le fichier \cmd{.bash\_login}. Ces
instructions ne sont donc exécutées qu'une seule fois.

Les commandes du fichier \cmd{.bashrc} sont exécutées chaque fois qu'un
interprète \textsc{shell} \cmd{bash} est lancé (ouverture d'un terminal).

Lors de la déconnexion c'est le fichier \cmd{.bash\_logout} qui est pris en compte.

Dans ces fichiers, on peut :

\begin{itemize}
  \item définir des alias,
  \item modifier des chemins d'accès,
  \item configurer des programmes,
  \item \ldots
\end{itemize}
Pour prendre en compte les modifications effectuées dans les fichiers de
configuration, vous pouvez :

\begin{itemize}
  \item soit vous déconnecter puis vous reconnecter sur la machine, dans le
        cas des fichiers \cmd{.bash\_logout} et \cmd{.bash\_login},
  \item soit lancer un nouveau \textsc{shell} (commande \cmd{bash}),
  \item soit utiliser la commande \cmd{source} qui exécute les instructions
        contenues dans un fichier (\cmd{source .bash\_login} ou \cmd{source
        .bashrc}).
\end{itemize}

\

\begin{maw}
\begin{itemize}
  \item éditer le fichier \cmd{.bashrc} pour introduire des alias ou
        modifier la variable prompt,
  \item exécuter la commande \cmd{source .bashrc} et constater les
        changements,
  \item faire la même constatation dans tout nouveau terminal.
\end{itemize}
\end{maw}

\newpage

\section{Deux commandes \shell{} bien utiles (niveau $**$)}
\label{grep}

\cmd{grep [options] exp [f1 ...]}
\begin{quote}
  Affiche toutes les lignes des fichiers f1...  contenant l'expression
  régulière \cmd{exp}. Si la commande porte sur plusieurs fichiers, chaque
  ligne affichée est précédée du nom de fichier la contenant.

  Quelques options intéressantes :
  \begin{itemize}
    \item[$\bullet$] \verb+ -i + rend la recherche insensible à la «casse" des
                     lettres (minuscules ou MAJUSCULES)
    \item[$\bullet$] \verb+ -n + indique la ligne où l'expression a été
                     trouvée
    \item[$\bullet$] \verb+ -l + ne donne que le nom des fichiers où la
                     recherche a été fructueuse (utile par exemple pour éditer
                     les fichiers qui contiennent l'expression recherchée)
  \end{itemize}

\begin{verbatim}
  grep alias $HOME/.bashrc
  grep begin *.tex
  grep -in The tolkien.txt
\end{verbatim}

\paragraph{Remarques :} 
\begin{itemize}
  \item il convient, bien entendu, se placer dans un répertoire
        qui contient les « bons » fichiers,
  \item la variable globale \verb+HOME+ (que nous reverrons au dernier TP)
        contient le chemin de votre répertoire de connexion.
  \item \texttt{exp} peut contenir des métacaractères permettant de décrire
        des motifs de recherche. Ces métacaractères sont spécifiques à la
        commande \texttt{grep} et sont différents de ceux du \shell.
\end{itemize}
%Ces exemples ne montrent pas toutes les possibilités de cette commande ;
%nous vous rappelons que pour des informations supplémentaires, il est très
%utile d'utiliser la commande \cmd{man}.
\end{quote}

\cmd{find repertoires expression...}
\begin{quote}
  Parcourt récursivement les fichiers des répertoires spécifiés et évalue,
  pour chaque fichier l'expression de gauche à droite.  L'expression permet
  soit de définir des critères de sélection, soit d'effectuer des opérations
  (affichages, exécution de commandes) sur les fichiers vérifiant les critères
  de sélection précédents.

\begin{verbatim}
  find . -name '*.txt' -print
\end{verbatim}
affiche (option \cmd{-print}) tous les fichiers de l'arborescence du répertoire
courant (\textbf{.}) qui ont comme suffixe \cmd{*.txt} (option \cmd{-name
'*.txt'}). 

\begin{verbatim}
  find $HOME -name '*.txt' -print -exec grep -i '^the'  {} \;
\end{verbatim}
trouve tous les fichiers de l'arborescence de votre répertoire de connexion
(\$HOME) dont le suffixe est \cmd{*.txt} et pour ces fichiers, affiche leur
nom puis, affiche les lignes contenant l'expression \cmd{the} en début de
ligne (option \cmd{-exec} avec la commande \cmd{grep -i '\^{}the'}) et s'en va
faire le café.

\paragraph{Remarques :}

\begin{itemize}
  \item "\{\}" désigne le nom des fichiers trouvés précédemment par le
        \cmd{find},
  \item le ";" (précédé de son \verb+\+) marque la fin du bloc \cmd{-exec},
  \item dans le dernier exemple, les affichages du \cmd{-print} et celui du
        \cmd{grep} sont effectués à la suite l'un de l'autre.
\end{itemize}
\end{quote}

Une dernière utilisation de la commande \cmd{find} avec en prime un tube et la
commande \cmd{xargs} :

\begin{verbatim}
  find $HOME -name '*.txt' -print | xargs grep -i '^the'
\end{verbatim}

Cette commande donne le même résultat que la précédente et on se passe des
parenthèses et de l'antislash-point virgule. Elle fonctionne comme suit :

\begin{itemize}
  \item le flux résultat de find est en entrée de la commande \cmd{xargs},
  \item celle-ci exécute autant de fois la commande en argument (ici 
        \cmd{grep}) qu'il y a d'éléments en entrée,
  \item chacun de ces éléments devenant le dernier argument de la commande
        \cmd{grep}.
\end{itemize}

\section{CQFAR (Ce Qu'il Faut Avoir Retenu)}

À la fin de cette séance de TP vous devez:
\begin{enumerate}
  \item savoir récupérer une archive,
  \item savoir archiver et désarchiver le contenu d'un répertoire ou un
        ensemble de fichiers,
  \item savoir copier, déplacer, renommer, supprimer des fichiers et des
        répertoires,
  \item savoir compresser et décompresser un fichier,
  \item savoir utiliser les caractères jokers,
%  \item savoir rediriger la sortie standard,
%  \item savoir coupler des commandes au moyen de tubes,
  \item savoir modifier une variable locale du \shell, et créer des
        \verb!alias!,
  \item savoir modifier le fichier \verb!.bashrc! et prendre
        en compte ces modifications.
%  \item savoir utiliser \verb!grep! et  \verb!find! dans des cas simples.
\end{enumerate}

\end{document}
